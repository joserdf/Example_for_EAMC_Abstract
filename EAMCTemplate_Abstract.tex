%% Esse arquivo foi produzido com base no arquivo FCEFyN-paper.tex, template das publicações da "Revista de la Facultad de Ciencias Exactas, Físicas y Naturales
%% de la Universidad Nacional de Córdoba, Argentina.

%% Originalmente feito por Ana Luiza Martins Karl, 17/06/2019

%% Revisado por Matheus Muller Pereira da Silva, 18/07/2019

%% Atualizado por José Renato Duarte Fajardo 15/08/2025

%% Em caso de problemas, envie um e-mail para eamc@lncc.br
\documentclass{eamc-class}


% pacote para gerar "dummy text", 
% comente para inserir seu texto
\usepackage{lipsum} 

% Título do trabalho
\title{Example for EAMC Abstract}

% Título curto para cabeçalho
\shorttitle{Example for EAMC Abstract}
       
% Autores
\author[1]{Full name of first author}
\author[2]{Full name of second author}
\author[1,2]{Full name of third author}

% Instituições 
\affil[1]{Institute A, Petrópolis/RJ, Brazil}
\affil[2]{Institute B, Petrópolis/RJ, Brazil}


% Sobrenome do primeiro autor para cabeçalho
\firstauthor{Last name of first author}

% Contato do autor para serem disponibilizados
\contactauthor{Name}
\email{email@email.com}  %email

% Dados da publicação
\thisvolume{Anais do XIX EAMC}
\thisyear{2026}

% Inicio do documento
\begin{document}


% Resumo e palavras chave
\abstract{This paper may be written in English or Portuguese. It should contain the main objectives, methodology and expected results of the
submitted work. Figures or equations are allowed in the abstract. References should be cited in the text using, for example, \cite{ArslanHansen1996} or \citep{ArslanHansen1996}.}

\keywords{
First keyword, Second keyword, Third keyword (up to 5 keywords)}

% Inclui titulo, resumo e palavras chave
\maketitle
\thispagestyle{fancy}
\printcontactdata

% Referencias
\insertbibliography{eamc-ref.bib}


\end{document}
